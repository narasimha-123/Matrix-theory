\documentclass[journal,12pt,twocolumn]{IEEEtran}

\usepackage{setspace}
\usepackage{gensymb}
\singlespacing
\usepackage[cmex10]{amsmath}

\usepackage{amsthm}

\usepackage{mathrsfs}
\usepackage{txfonts}
\usepackage{stfloats}
\usepackage{bm}
\usepackage{cite}
\usepackage{cases}
\usepackage{subfig}

\usepackage{longtable}
\usepackage{multirow}

\usepackage{enumitem}
\usepackage{mathtools}
\usepackage{steinmetz}
\usepackage{tikz}
\usepackage{circuitikz}
\usepackage{verbatim}
\usepackage{tfrupee}
\usepackage[breaklinks=true]{hyperref}
\usepackage{graphicx}
\usepackage{tkz-euclide}

\usetikzlibrary{calc,math}
\usepackage{listings}
    \usepackage{color}                                            %%
    \usepackage{array}                                            %%
    \usepackage{longtable}                                        %%
    \usepackage{calc}                                             %%
    \usepackage{multirow}                                         %%
    \usepackage{hhline}                                           %%
    \usepackage{ifthen}                                           %%
    \usepackage{lscape}     
\usepackage{multicol}
\usepackage{chngcntr}

\DeclareMathOperator*{\Res}{Res}

\renewcommand\thesection{\arabic{section}}
\renewcommand\thesubsection{\thesection.\arabic{subsection}}
\renewcommand\thesubsubsection{\thesubsection.\arabic{subsubsection}}

\renewcommand\thesectiondis{\arabic{section}}
\renewcommand\thesubsectiondis{\thesectiondis.\arabic{subsection}}
\renewcommand\thesubsubsectiondis{\thesubsectiondis.\arabic{subsubsection}}


\hyphenation{op-tical net-works semi-conduc-tor}
\def\inputGnumericTable{}                                 %%

\lstset{
%language=C,
frame=single, 
breaklines=true,
columns=fullflexible
}
\begin{document}


\newtheorem{theorem}{Theorem}[section]
\newtheorem{problem}{Problem}
\newtheorem{proposition}{Proposition}[section]
\newtheorem{lemma}{Lemma}[section]
\newtheorem{corollary}[theorem]{Corollary}
\newtheorem{example}{Example}[section]
\newtheorem{definition}[problem]{Definition}

\newcommand{\BEQA}{\begin{eqnarray}}
\newcommand{\EEQA}{\end{eqnarray}}
\newcommand{\define}{\stackrel{\triangle}{=}}
\bibliographystyle{IEEEtran}
\raggedbottom
\setlength{\parindent}{0pt}
\providecommand{\mbf}{\mathbf}
\providecommand{\pr}[1]{\ensuremath{\Pr\left(#1\right)}}
\providecommand{\qfunc}[1]{\ensuremath{Q\left(#1\right)}}
\providecommand{\sbrak}[1]{\ensuremath{{}\left[#1\right]}}
\providecommand{\lsbrak}[1]{\ensuremath{{}\left[#1\right.}}
\providecommand{\rsbrak}[1]{\ensuremath{{}\left.#1\right]}}
\providecommand{\brak}[1]{\ensuremath{\left(#1\right)}}
\providecommand{\lbrak}[1]{\ensuremath{\left(#1\right.}}
\providecommand{\rbrak}[1]{\ensuremath{\left.#1\right)}}
\providecommand{\cbrak}[1]{\ensuremath{\left\{#1\right\}}}
\providecommand{\lcbrak}[1]{\ensuremath{\left\{#1\right.}}
\providecommand{\rcbrak}[1]{\ensuremath{\left.#1\right\}}}
\theoremstyle{remark}
\newtheorem{rem}{Remark}
\newcommand{\sgn}{\mathop{\mathrm{sgn}}}
\providecommand{\abs}[1]{\left\vert#1\right\vert}
\providecommand{\res}[1]{\Res\displaylimits_{#1}} 
\providecommand{\norm}[1]{\left\lVert#1\right\rVert}
%\providecommand{\norm}[1]{\lVert#1\rVert}
\providecommand{\mtx}[1]{\mathbf{#1}}
\providecommand{\mean}[1]{E\left[ #1 \right]}
\providecommand{\fourier}{\overset{\mathcal{F}}{ \rightleftharpoons}}
%\providecommand{\hilbert}{\overset{\mathcal{H}}{ \rightleftharpoons}}
\providecommand{\system}{\overset{\mathcal{H}}{ \longleftrightarrow}}
	%\newcommand{\solution}[2]{\textbf{Solution:}{#1}}
\newcommand{\solution}{\noindent \textbf{Solution: }}
\newcommand{\cosec}{\,\text{cosec}\,}
\providecommand{\dec}[2]{\ensuremath{\overset{#1}{\underset{#2}{\gtrless}}}}
\newcommand{\myvec}[1]{\ensuremath{\begin{pmatrix}#1\end{pmatrix}}}
\newcommand{\mydet}[1]{\ensuremath{\begin{vmatrix}#1\end{vmatrix}}}
\numberwithin{equation}{subsection}

\makeatletter
\@addtoreset{figure}{problem}
\makeatother
\let\StandardTheFigure\thefigure
\let\vec\mathbf

\renewcommand{\thefigure}{\theproblem}

\def\putbox#1#2#3{\makebox[0in][l]{\makebox[#1][l]{}\raisebox{\baselineskip}[0in][0in]{\raisebox{#2}[0in][0in]{#3}}}}
     \def\rightbox#1{\makebox[0in][r]{#1}}
     \def\centbox#1{\makebox[0in]{#1}}
     \def\topbox#1{\raisebox{-\baselineskip}[0in][0in]{#1}}
     \def\midbox#1{\raisebox{-0.5\baselineskip}[0in][0in]{#1}}
\vspace{3cm}
\title{Assignment 8}
\author{Sachinkumar Dubey - EE20MTECH11009}
\maketitle
\newpage
\bigskip
\renewcommand{\thefigure}{\theenumi}
\renewcommand{\thetable}{\theenumi}
Download all python codes from 
\begin{lstlisting}
https://github.com/sachinomdubey/Matrix-theory/Assignment8/codes
\end{lstlisting}
%
and latex-tikz codes from 
%
\begin{lstlisting}
https://github.com/sachinomdubey/Matrix-theory/Assignment8
\end{lstlisting}
\section{Problem}
(Dresden/Page80/Example1/D)\\Determine the distance of the point $D(-1, 2, -4)$ from the plane given below. Also find the foot of perpendicular drawn from the point D to the given plane using SVD.
\begin{align}
3x+2y-6z-2=0\label{eq:1.0.1}
\end{align}
\section{Solution}
Equation of plane can be written in the form: 
\begin{align}\label{eq1}
	\vec{n}^T\vec{x} &= c
\end{align}
Writing the given plane equation \eqref{eq:1.0.1} in the form \eqref{eq1}:
\begin{align}\label{eq2}
	\myvec{3 & 2 & -6}\myvec{x\\y\\z} &= 2
\intertext{Where,}
    \vec{n} &= \myvec{3\\2\\-6} \\
    \vec{x} &= \myvec{x\\y\\z}  \quad  c = 2
\end{align}
A vector from the plane to the point $D(-1, 2, -4)$ is given by:
\begin{align}
\vec{w}=\vec{d}-\vec{x}
\intertext{Where,}
    \vec{d} &= \myvec{-1\\2\\-4} \\
    \vec{x} &= \myvec{x\\y\\z} 
\end{align}
The projection of $\vec{w}$ onto the normal vector $\vec{n}$ can be written as:
\begin{align}
    \text{proj}_\vec{n}\vec{w}=\frac{\vec{n}^T\vec{w}}{\vec{n}^T\vec{n}}\cdot\vec{n}\\
    \text{proj}_\vec{n}\vec{w}=\frac{\vec{n}^T(\vec{d}-\vec{x})}{\vec{n}^T\vec{n}}\cdot\vec{n}\\
     \text{proj}_\vec{n}\vec{w}=\frac{\vec{n}^T\vec{d}-\vec{n}^T\vec{x}}{\vec{n}^T\vec{n}}\cdot\vec{n}\\
\end{align}
From equation \eqref{eq1},
\begin{align}
     \text{proj}_\vec{n}\vec{w}=\frac{\vec{n}^T\vec{d}-c}{\vec{n}^T\vec{n}}\cdot\vec{n}\\
\end{align}
Putting the values of $\vec{n}$, $\vec{d}$ and $c$, we get:
\begin{align}
     \text{proj}_\vec{n}\vec{w}=\frac{\myvec{3&2&-6}\myvec{-1\\2\\-4} -2}{\myvec{3&2&-6}\myvec{3\\2\\-6}}\cdot\myvec{3\\2\\-6}\\
      \text{proj}_\vec{n}\vec{w}=\frac{23}{49}\cdot\myvec{3\\2\\-6}
\end{align}
The distance $d_{min}$ between point $D(-1, 2, -4)$ and the given plane  is obtained as:
\begin{align}
    d_{min}=\norm{\text{proj}_\vec{n}\vec{w}}\\
    d_{min}=\frac{23}{49}\cdot\norm{\myvec{3\\2\\-6}}
\end{align}
\begin{align}
    \therefore d_{min}=\frac{23}{49} \times \sqrt{(3)^2+(2)^2+(-6)^2}\\
    \therefore d_{min}=\frac{23}{49} \times 7\\
    \implies \boxed{d_{min}=\frac{23}{7}= 3.2857} \label{eq_20}
\end{align}
\begin{figure}[h!]
\centering
\resizebox{\columnwidth}{!}
    {
    \input{Figure.tex}
    }
    \caption{Distance between a point and a plane}
\end{figure}
\\
\textbf{Finding the foot of perpendicular from point $\vec{D}$ to the given plane using SVD:}\\
We need to represent the equation of plane in parametric form,
\begin{align}
	\vec{Q} = \vec{p} + \alpha_1\vec{m_1} + \alpha_2\vec{m_2}\label{eq3}
\end{align}
Here $\vec{p}$ is any point on plane and $\vec{m_1}, \vec{m_2}$ are two vectors parallel to plane and hence $\perp$ to $\vec{n}$.
First we find orthogonal vectors $\vec{m_1}$ and $\vec{m_2}$ to the vector $\vec{n}$. Let, $\vec{m} = \myvec{a\\b\\c}$, then
\begin{align}
\vec{m^T}\vec{n} &= 0 \nonumber \\
\implies \myvec{a & b & c} \myvec{3 \\ 2 \\ -6} &= 0 \nonumber \\
\implies 3a+2b-6c &= 0 \label{eq:eq_1}
\end{align}
By substituting $a=1;b=0$ in \eqref{eq:eq_1},
\begin{align} \label{eq:eq_2}
    \vec{m_1} = \myvec{1 \\ 0 \\ 1/2} 
\end{align}
By substituting $a=0;b=1$ in \eqref{eq:eq_1},
\begin{align} \label{eq:eq_3}
    \vec{m_2} = \myvec{0 \\ 1 \\ 1/3} 
\end{align}
Let us find point $\vec{p}$ on the plane. Put $x=1,z=0$ in \eqref{eq2}, we get $\vec{p} = \myvec{1\\\frac{-1}{2}\\0}$\\
Let $\vec{Q}$ be the point on plane with shortest distance to point $\vec{d}$.
$\vec{Q}$ can be expressed in \eqref{eq3} form as
\begin{align}\label{eq6}
	\vec{Q} = \myvec{1\\\frac{-1}{2}\\0} + \alpha_1\myvec{1\\0\\\frac{1}{2}} + \alpha_2\myvec{0\\1\\\frac{1}{3}} \label{eq_25}
\end{align}
Computation of Pseudo Inverse using SVD in order to determine the value of $\alpha_1$ and $\alpha_2$ :
\begin{align}
	\label{eq7}\myvec{1\\\frac{-1}{2}\\0} + \alpha_1\myvec{1\\0\\\frac{1}{2}} + \alpha_2\myvec{0\\1\\\frac{1}{3}} &= \myvec{-1\\2\\-4}\\\label{eq8}
	\alpha_1\myvec{1\\0\\\frac{1}{2}} + \alpha_2\myvec{0\\1\\\frac{1}{3}}  &= \myvec{-2\\\frac{5}{2}\\-4}\\\label{eq9}
	\myvec{1 & 0\\0 & 1\\\frac{1}{2} & \frac{1}{3}} \myvec{\alpha_1 \\ \alpha_2} &=\myvec{-2\\\frac{5}{2}\\-4}\\\label{eq10}
	\vec{M}\vec{x} &= \vec{b}\label{eq11}
\intertext{Where,}
    \vec{M} &= \myvec{1 & 0\\0 & 1\\\frac{1}{2} & \frac{1}{3}} \\
    \vec{x} &= \myvec{\alpha_1 \\ \alpha_2} \\
    \vec{b} &= \myvec{-2\\\frac{5}{2}\\-4}
\end{align}
Applying Singular Value Decomposition on $\vec{M}$,
\begin{align} \label{eq:eq_6}
    \vec{M}=\vec{U}\vec{S}\vec{V}^T
\end{align}
Where the columns of $\vec{V}$ are the eigenvectors of $\vec{M}^T\vec{M}$, the columns of $\vec{U}$ are the eigenvectors of $\vec{M}\vec{M}^T$ and $\vec{S}$ is diagonal matrix of singular values of $\vec{M}^T\vec{M}$.
\begin{align}
    \vec{M}^T \vec{M} &= \myvec{\tfrac{5}{4} & \tfrac{1}{6}  \\\tfrac{1}{6}  & \tfrac{10}{9} }\label{eq:eq_7} \\
    \vec{M} \vec{M}^T &= \myvec{1 & 0 & \frac{1}{2}\\ 0 & 1 & \frac{1}{3} \\ \tfrac{1}{2} & \tfrac{1}{3} & \tfrac{13}{36}} \label{eq:eq_8}
\end{align}
From \eqref{eq11} and \eqref{eq:eq_6},
\begin{align}
    \vec{U} \vec{S} \vec{V}^T \vec{x} = \vec{b} \nonumber \\
    \implies \vec{x} = \vec{V} \vec{S_+} \vec{U^T} \vec{b} \label{eq:eq_9}
\end{align}
Where $\vec{S_+}$ is Moore-Penrose Pseudo-Inverse of $\vec{S}$. Calculating eigenvalues of $\vec{M}\vec{M}^T$,
\begin{align}
    \mydet{\vec{M} \vec{M}^T - \lambda \vec{I}} = 0 \nonumber \\
    \implies \mydet{1-\lambda & 0 & \frac{1}{2} \\ 0 & 1-\lambda & \frac{1}{3} \\ \frac{1}{2} & \frac{1}{3} & \frac{13}{36}-\lambda} &= 0 \nonumber \\
    \implies \lambda^3 - \frac{85}{36}\lambda^2 + \frac{49}{36}\lambda =0 \nonumber
\end{align}
Hence eigenvalues of $\vec{M}\vec{M}^T$ are,
\begin{align} \label{eq:eq_10}
    \lambda_1 = \frac{49}{36}; \quad \lambda_2 = 1; \quad \lambda_3 =0
\end{align}
And the corresponding eigenvectors are,
\begin{align}
     \vec{u_1} = \myvec{18 \\12 \\ 13}; \quad \vec{u_2} = \myvec{-2 \\3 \\ 0};\quad 
    \vec{u_3} = \myvec{-3 \\ -2 \\ 6}\label{eq:eq_11} 
\end{align}
Normalizing the above eigenvectors,
\begin{align} 
     \vec{u_1} = \myvec{\tfrac{18}{7\sqrt{13}} \\\tfrac{12}{7\sqrt{13}}\\ \tfrac{13}{7\sqrt{13}}}; \quad 
    \vec{u_2} = \myvec{\frac{-2}{\sqrt{13}} \\\frac{3}{\sqrt{13}} \\ 0}; \quad 
    \vec{u_3} = \myvec{\tfrac{-3}{7} \\ \tfrac{-2}{7} \\ \tfrac{6}{7}}
  \label{eq:eq_12}
\end{align}
From \eqref{eq:eq_12} we obtain $\vec{U}$ as,
\begin{align} \label{eq:eq_13}
    \vec{U} = \myvec{\tfrac{18}{7\sqrt{13}} & \frac{-2}{\sqrt{13}} &\tfrac{-3}{7} \\ \tfrac{12}{7\sqrt{13}} & \frac{3}{\sqrt{13}} & \tfrac{-2}{7}  \\  \tfrac{13}{7\sqrt{13}} & 0 & \tfrac{6}{7}}
\end{align}
Using values from \eqref{eq:eq_10},
\begin{align} \label{eq:eq_14}
    \vec{S} = \myvec{\frac{7}{6} & 0 \\ 0 & 1 \\ 0 & 0} 
\end{align}
Calculating the eigenvalues of $\vec{M}^T\vec{M}$,
\begin{align}
    \mydet{\vec{M}^T\vec{M} - \lambda \vec{I}} = 0 \nonumber \\
    \implies \mydet{\tfrac{5}{4}-\lambda & \frac{1}{6} \\ \frac{1}{6} & \tfrac{10}{9}-\lambda} &= 0 \nonumber \\
    \implies \lambda^2 - \frac{85}{36}\lambda + \frac{49}{36} =0 \nonumber
\end{align}
Hence, eigenvalues of $\vec{M}^T\vec{M}$ are,
\begin{align}
    \lambda_4 = \frac{49}{36}; \quad \lambda_5 = 1 \nonumber
\end{align}
And the corresponding eigenvectors are,
\begin{align}
    \vec{v_1} = \myvec{3 \\2 }; \quad \vec{v_2} = \myvec{-2 \\3};\nonumber
    \intertext{Normalizing the above eigenvectors,}
     \vec{v_1} = \myvec{\frac{3}{\sqrt{13}}  \\ \frac{2}{\sqrt{13}}}; \quad 
    \vec{v_2} = \myvec{\frac{-2}{\sqrt{13}}\\\frac{3}{\sqrt{13}}} \label{eq:eq_15}
\end{align}
From\eqref{eq:eq_15} we obtain $\vec{V}$ as,
\begin{align} \label{eq:eq_16}
    \vec{V} = \myvec{\frac{3}{\sqrt{13}} & \frac{-2}{\sqrt{13}}\\ \frac{2}{\sqrt{13}} & \frac{3}{\sqrt{13}}}
\end{align}
From \eqref{eq:eq_6} we get the Singular Value Decomposition of $\vec{M}$,
\begin{align} \label{eq:eq_17}
    \vec{M} =  \myvec{\tfrac{18}{7\sqrt{13}} & \frac{-2}{\sqrt{13}} &\tfrac{-3}{7} \\ \tfrac{12}{7\sqrt{13}} & \frac{3}{\sqrt{13}} & \tfrac{-2}{7}  \\  \tfrac{13}{7\sqrt{13}} & 0 & \tfrac{6}{7}} 
    \myvec{\frac{7}{6} & 0 \\ 0 & 1 \\ 0 & 0}  \myvec{\frac{3}{\sqrt{13}} & \frac{-2}{\sqrt{13}}\\ \frac{2}{\sqrt{13}} & \frac{3}{\sqrt{13}}}
\end{align}
Moore-Penrose Pseudo inverse of $\vec{S}$ is given by,
\begin{align} \label{eq:eq_18}
    \vec{S_+} =\myvec{\frac{6}{7} & 0 & 0 \\ 0 & 1 &0 }
\end{align}
From \eqref{eq:eq_9},
\begin{align}
    \vec{U}^T\vec{b} &= \myvec{\tfrac{-58}{7\sqrt{13}} \\ 
    \tfrac{23}{2\sqrt{13}} \\ \tfrac{-23}{7}} \nonumber \\
    \vec{S_+}\vec{U}^T\vec{d} &= \myvec{\frac{-348}{49\sqrt{13}} \\ \frac{23}{2\sqrt{13}}} \nonumber \\
    \vec{x} = \vec{V}\vec{S_+}\vec{U}^T\vec{d} &= \boxed{\myvec{\frac{-167}{49} \\ \frac{153}{98}}} \label{eq:eq_19}
\end{align}
To verify the value of $\vec{x}$ obtained from \eqref{eq:eq_19},
\begin{align} \label{eq:eq_20}
    \vec{M}^T\vec{M}\vec{x} = \vec{M}^T\vec{b}
\end{align}
Substituting the values from \eqref{eq:eq_7} in \eqref{eq:eq_20},
\begin{align}
    \myvec{\frac{5}{4} & \frac{1}{6} \\\frac{1}{6} & \frac{10}{9}}\vec{x} = \myvec{-4 \\ \frac{7}{6}} \nonumber
\end{align}
Converting the above equation into augmented form and solving for $\vec{x}$,
\begin{align}
    \myvec{\frac{5}{4} & \frac{1}{6} & -4\\ \frac{1}{6} & \frac{10}{9} &\frac{7}{6}}\nonumber\\ 
    \xleftrightarrow[]{R_1 \leftarrow \frac{4}{5}R_1} 
    \myvec{1 & \frac{2}{15} & \frac{-16}{5} \\ \frac{1}{6} & \frac{10}{9} & \frac{7}{6}} \nonumber \\ 
    \xleftrightarrow[]{R_2 \leftarrow R_2-\frac{1}{6}R_1} \myvec{1 & \frac{2}{15} & \frac{-12}{5} \\ 0 & \frac{49}{45} & \frac{17}{10}}\nonumber\\
    \xleftrightarrow[]{R_2 \leftarrow \frac{45}{49}R_2} \myvec{1 & \frac{2}{15} & \frac{-12}{5} \\ 0 & 1 & \frac{153}{98}}\nonumber\\
    \xleftrightarrow[]{R_1 \leftarrow R_1-\frac{2}{15}R_2} \myvec{1 & 0 & \frac{-167}{49} \\ 0 & 1 & \frac{153}{98}}\label{eq:eq_22}
\end{align}
From \eqref{eq:eq_22} it can be observed that,
\begin{align} 
    \vec{x} = \myvec{\frac{-167}{49} \\ \frac{153}{98}} 
\end{align}
Hence verified.\\
Thus, the point $\vec{Q}$ (foot of the perpendicular) can be obtained by putting values of $\alpha_1$ and $\alpha_2$ in \eqref{eq_25}:
\begin{align}
    \vec{Q} = \myvec{1\\\frac{-1}{2}\\0} + \frac{-167}{49}\myvec{1\\0\\\frac{1}{2}} +  \frac{153}{98}\myvec{0\\1\\\frac{1}{3}}\\
    \boxed{\vec{Q} = \myvec{\tfrac{-118}{49}\\\tfrac{52}{49}\\\tfrac{-58}{49}}}
\end{align}
The distance between the point $D$ and the plane can be obtained as:
\begin{align}
    \norm{\vec{Q}-\vec{d}}=\sqrt{\brak{\tfrac{-118}{49}+1}^2+\brak{\tfrac{52}{49}-2}^2+\brak{\tfrac{-58}{49}+4}^2}\\
    \boxed{\norm{\vec{Q}-\vec{d}}=\frac{23}{7}=3.2857} \label{eq_54}
\end{align}
Thus the distance obtained in equation \eqref{eq_20} by projection method is matches with the distance obtained in equation \eqref{eq_54}. Hence verified.
\end{document}