\documentclass[journal,12pt,twocolumn]{IEEEtran}
\usepackage{longtable}
\usepackage{setspace}
\usepackage{gensymb}
\singlespacing
\usepackage[cmex10]{amsmath}
\newcommand\myemptypage{
	\null
	\thispagestyle{empty}
	\addtocounter{page}{-1}
	\newpage
}
\usepackage{amsthm}
\usepackage{mdframed}
\usepackage{mathrsfs}
\usepackage{txfonts}
\usepackage{stfloats}
\usepackage{bm}
\usepackage{cite}
\usepackage{cases}
\usepackage{subfig}

\usepackage{longtable}
\usepackage{multirow}

\usepackage{enumitem}
\usepackage{mathtools}
\usepackage{steinmetz}
\usepackage{tikz}
\usepackage{circuitikz}
\usepackage{verbatim}
\usepackage{tfrupee}
\usepackage[breaklinks=true]{hyperref}
\usepackage{graphicx}
\usepackage{tkz-euclide}

\usetikzlibrary{calc,math}
\usepackage{listings}
    \usepackage{color}                                            %%
    \usepackage{array}                                            %%
    \usepackage{longtable}                                        %%
    \usepackage{calc}                                             %%
    \usepackage{multirow}                                         %%
    \usepackage{hhline}                                           %%
    \usepackage{ifthen}                                           %%
    \usepackage{lscape}     
\usepackage{multicol}
\usepackage{chngcntr}

\DeclareMathOperator*{\Res}{Res}

\renewcommand\thesection{\arabic{section}}
\renewcommand\thesubsection{\thesection.\arabic{subsection}}
\renewcommand\thesubsubsection{\thesubsection.\arabic{subsubsection}}

\renewcommand\thesectiondis{\arabic{section}}
\renewcommand\thesubsectiondis{\thesectiondis.\arabic{subsection}}
\renewcommand\thesubsubsectiondis{\thesubsectiondis.\arabic{subsubsection}}


\hyphenation{op-tical net-works semi-conduc-tor}
\def\inputGnumericTable{}                                 %%

\lstset{
%language=C,
frame=single, 
breaklines=true,
columns=fullflexible
}
\begin{document}


\newtheorem{theorem}{Theorem}[section]
\newtheorem{problem}{Problem}
\newtheorem{proposition}{Proposition}[section]
\newtheorem{lemma}{Lemma}[section]
\newtheorem{corollary}[theorem]{Corollary}
\newtheorem{example}{Example}[section]
\newtheorem{definition}[problem]{Definition}

\newcommand{\BEQA}{\begin{eqnarray}}
\newcommand{\EEQA}{\end{eqnarray}}
\newcommand{\define}{\stackrel{\triangle}{=}}
\bibliographystyle{IEEEtran}
\raggedbottom
\setlength{\parindent}{0pt}
\providecommand{\mbf}{\mathbf}
\providecommand{\pr}[1]{\ensuremath{\Pr\left(#1\right)}}
\providecommand{\qfunc}[1]{\ensuremath{Q\left(#1\right)}}
\providecommand{\sbrak}[1]{\ensuremath{{}\left[#1\right]}}
\providecommand{\lsbrak}[1]{\ensuremath{{}\left[#1\right.}}
\providecommand{\rsbrak}[1]{\ensuremath{{}\left.#1\right]}}
\providecommand{\brak}[1]{\ensuremath{\left(#1\right)}}
\providecommand{\lbrak}[1]{\ensuremath{\left(#1\right.}}
\providecommand{\rbrak}[1]{\ensuremath{\left.#1\right)}}
\providecommand{\cbrak}[1]{\ensuremath{\left\{#1\right\}}}
\providecommand{\lcbrak}[1]{\ensuremath{\left\{#1\right.}}
\providecommand{\rcbrak}[1]{\ensuremath{\left.#1\right\}}}
\theoremstyle{remark}
\newtheorem{rem}{Remark}
\newcommand{\sgn}{\mathop{\mathrm{sgn}}}
\providecommand{\abs}[1]{\left\vert#1\right\vert}
\providecommand{\res}[1]{\Res\displaylimits_{#1}} 
\providecommand{\norm}[1]{\left\lVert#1\right\rVert}
%\providecommand{\norm}[1]{\lVert#1\rVert}
\providecommand{\mtx}[1]{\mathbf{#1}}
\providecommand{\mean}[1]{E\left[ #1 \right]}
\providecommand{\fourier}{\overset{\mathcal{F}}{ \rightleftharpoons}}
%\providecommand{\hilbert}{\overset{\mathcal{H}}{ \rightleftharpoons}}
\providecommand{\system}{\overset{\mathcal{H}}{ \longleftrightarrow}}
	%\newcommand{\solution}[2]{\textbf{Solution:}{#1}}
\newcommand{\solution}{\noindent \textbf{Solution: }}
\newcommand{\cosec}{\,\text{cosec}\,}
\providecommand{\dec}[2]{\ensuremath{\overset{#1}{\underset{#2}{\gtrless}}}}
\newcommand{\myvec}[1]{\ensuremath{\begin{pmatrix}#1\end{pmatrix}}}
\newcommand{\mydet}[1]{\ensuremath{\begin{vmatrix}#1\end{vmatrix}}}
\numberwithin{equation}{subsection}

\makeatletter
\@addtoreset{figure}{problem}
\makeatother
\let\StandardTheFigure\thefigure
\let\vec\mathbf

\renewcommand{\thefigure}{\theproblem}

\def\putbox#1#2#3{\makebox[0in][l]{\makebox[#1][l]{}\raisebox{\baselineskip}[0in][0in]{\raisebox{#2}[0in][0in]{#3}}}}
     \def\rightbox#1{\makebox[0in][r]{#1}}
     \def\centbox#1{\makebox[0in]{#1}}
     \def\topbox#1{\raisebox{-\baselineskip}[0in][0in]{#1}}
     \def\midbox#1{\raisebox{-0.5\baselineskip}[0in][0in]{#1}}
\vspace{3cm}
\title{Assignment 10}
\author{Sachinkumar Dubey - EE20MTECH11009}
\maketitle
\newpage
\bigskip
\renewcommand{\thefigure}{\theenumi}
\renewcommand{\thetable}{\theenumi}
%
Download the latex-tikz codes from 
%
\begin{lstlisting}
https://github.com/sachinomdubey/Matrix-theory/Assignment10
\end{lstlisting}
\section{Problem}
(Hoffman/Page123/8) : 
%
If $F$ is a field and $h$ is a polynomial over $F$ of degree $\geq 1$,  show that the mapping $f \rightarrow f(h)$ is a one-one linear transformation of $F[x]$ into $F[x]$. Show that this transformation is an isomorphism of $F[x]$ onto $F[x]$ if and only if deg $h = 1$. 
\section{Definition and Result used}
\begin{table}[h!]
	\begin{longtable}{|l|l|}
		\hline
		\multirow{3}{*}{Linear Transformation} 
		& \\
		& A Transformation $T(x)$ from $V \rightarrow W$ is said to be linear if it satisfies \\
		&the following:\\
		&\\
		& \qquad $T(ax+y)=aT(x)+T(y)$ \\
		& \qquad where, $x, y \in V$ and $a \in F$.\\
		& \\
		\hline
		\multirow{3}{*}{One to one transformation} & \\
		& A transformation $T(x)$ from $V\rightarrow W$ is one to one, if: \\
		& \qquad $T(x_1)=T(x_2) \implies x_1=x_2$ \\
	    & i.e. The mapping is unique.\\
	    & \\
	    \hline
	    \multirow{3}{*}{Isomorphism} & \\
		& A transformation $T(x)$ from $V\rightarrow W$ is isomorphism of $V$ onto $W$, if there \\
		& exists an inverse $T'(x)$ of $T(x)$ such that $T'T(x)=TT'(x)=x$\\
	    & \\
	    \hline
	    \multirow{3}{*}{Degree of  a polynomial} & \\
		& The degree of a polynomial $p(x)$ is the highest power of $x$ in the \\
	    & polynomial $p(x)$.\\
	    &\\
	    \hline
	\end{longtable}
\end{table}
\section{Solution}
\onecolumn
	\begin{longtable}{|l|l|}
		\hline
		\multirow{3}{*}{Proving that the given} 
		& \\
		& Here, $F[x]$ is a set of polynomials over field $F$, written as:\\ mapping is linear.
		&\\
		& \qquad \qquad \qquad $F[x]=\left \{\sum_{i=0}^\infty a_ix^i\quad \mid \quad a_i\in F\right \}$\\
		&\\
		& Let the mapping $f \rightarrow f(h)$ represented as: \\
		&\\
		&\qquad \qquad \qquad $G(f)=f(h)$\\
		&\\
		& Thus, $G(f)$ is clearly a function from $F[x]$ into $F[x]$.\\
		& Let, $f,g \in F[x]$ and $\alpha \in F$\\
		&\\
		& \qquad \qquad \qquad $G(\alpha f+g)=(\alpha f+g)(h)$ \\
        & \qquad \qquad \qquad \qquad\qquad$=(\alpha f)(h)+g(h)$ \\
        & \qquad \qquad \qquad \qquad\qquad$=\alpha f(h)+g(h)$ \\
        & \qquad \qquad \qquad \qquad\qquad$=\alpha G(f)+G(g)$\\
        &\\
        &Thus, $G$ is a linear transformation\\
        &\\
		\hline
		\multirow{3}{*}{Proving that $G$ is} & \\
		& Now to show that $G$ is one to one,\\one-one transformation
		&\\
		& \qquad \qquad \qquad $G(f)=G(g)$ \\
	    & \qquad \qquad \qquad $\therefore f(h)=g(h)$\\
	    & \qquad \qquad $\sum_{i=0}^n c_i[h(x)]^i=\sum_{j=0}^m c_j[h(x)]^j$\\
	    &\\
	    &Here, $h(x)$ is parameter to both $f$ and $g$, hence $i=j$ and $c_i=c_j$. Thus,\\
	    &\\
	    & \qquad \qquad \qquad $f=g$\\
	    &\\
	    &Therefore, $G$ is one-one linear transformation \\
	    &\\
	    \hline
	    \multirow{3}{*}{Proving $G$ is Isomorphism} & \\
		&  Let deg$(h)=1$, then $h$ can be written as:\\ if and only if deg $h=1$.
        &\\
        & \qquad \qquad \qquad $h=a+bx$,  Where, $b \ne 0$ \\
        &\\
        & Let us define $h'$ such that:\\
        &\\
        & \qquad \qquad \qquad $h'=\frac1bx-\frac ab$ \\
        &\\
        & Let $G'$ be the linear transformation from $F(x)$ to $F(x)$ given by:\\
        &\\
        & \qquad \qquad \qquad $G’(f)=f\brak{\frac1bx-\frac ab}$ \\
        &\\
        & It can be shown that $G'$ is inverse of $G$ as follow: \\
        & \qquad \qquad \qquad $G(G'(f))=G\brak{f\brak{\frac1bx-\frac ab}}$\\
        &\\
        \hline
        &\\
        & \qquad \qquad \qquad \qquad \qquad $=f\brak{a\brak{\frac1ax-\frac ba}+b}$\\
        & \qquad \qquad \qquad \qquad \qquad $=f(x)$\\
		& \\
        &Similarly,\\
        &\\ 
        & \qquad \qquad \qquad $G'(G(f))=G'\brak{f\brak{ax+b}}$\\
        & \qquad \qquad \qquad \qquad \qquad $=f\brak{\frac1a\brak{ax+b}-\frac ba}$\\
        & \qquad \qquad \qquad \qquad \qquad $=f(x)$\\
        &\\
        &Thus, $G'$ is inverse of $G$. Therefore, $G$ is isomorphism and we can say: \\
        &\\
        & \qquad \qquad \qquad $\boxed{\deg(h)=1 \implies \text{$G$ is isomorphism.}}$ \\
        &\\
        & Let $\deg(h)>1$, then \\
        &\\
        & \qquad \qquad \qquad $\deg f(h)=\deg f\cdot \deg h $\\
        & \qquad \qquad \qquad $\implies \deg f(h) \geq 1$\\
        & \qquad \qquad \qquad $\implies G(f)=f(h) \ne x$\\
        &This means the image of $G$ does not contain polynomials of degree one.\\
        &Hence $G$ is not onto and therefore $G$ can not be an isomorphism.\\
        &Thus we can write:\\
        &\\
        & \qquad \qquad \qquad $\boxed{\deg(h)>1 \implies \text{$G$ is not isomorphism.}}$ \\ 
        &\\
        &\\
        & Thus, We can conclude:\\
        &\\
        &\qquad \qquad \qquad $\boxed{\text{$G$ is isomorphism.} \iff \deg(h)=1} $\\
        &\\
	    \hline
    \end{longtable}
\end{document}