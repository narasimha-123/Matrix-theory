\documentclass[journal,12pt,twocolumn]{IEEEtran}

\usepackage{setspace}
\usepackage{gensymb}
\singlespacing
\usepackage[cmex10]{amsmath}

\usepackage{amsthm}
\usepackage{mdframed}
\usepackage{mathrsfs}
\usepackage{txfonts}
\usepackage{stfloats}
\usepackage{bm}
\usepackage{cite}
\usepackage{cases}
\usepackage{subfig}

\usepackage{longtable}
\usepackage{multirow}

\usepackage{enumitem}
\usepackage{mathtools}
\usepackage{steinmetz}
\usepackage{tikz}
\usepackage{circuitikz}
\usepackage{verbatim}
\usepackage{tfrupee}
\usepackage[breaklinks=true]{hyperref}
\usepackage{graphicx}
\usepackage{tkz-euclide}

\usetikzlibrary{calc,math}
\usepackage{listings}
    \usepackage{color}                                            %%
    \usepackage{array}                                            %%
    \usepackage{longtable}                                        %%
    \usepackage{calc}                                             %%
    \usepackage{multirow}                                         %%
    \usepackage{hhline}                                           %%
    \usepackage{ifthen}                                           %%
    \usepackage{lscape}     
\usepackage{multicol}
\usepackage{chngcntr}

\DeclareMathOperator*{\Res}{Res}

\renewcommand\thesection{\arabic{section}}
\renewcommand\thesubsection{\thesection.\arabic{subsection}}
\renewcommand\thesubsubsection{\thesubsection.\arabic{subsubsection}}

\renewcommand\thesectiondis{\arabic{section}}
\renewcommand\thesubsectiondis{\thesectiondis.\arabic{subsection}}
\renewcommand\thesubsubsectiondis{\thesubsectiondis.\arabic{subsubsection}}


\hyphenation{op-tical net-works semi-conduc-tor}
\def\inputGnumericTable{}                                 %%

\lstset{
%language=C,
frame=single, 
breaklines=true,
columns=fullflexible
}
\begin{document}


\newtheorem{theorem}{Theorem}[section]
\newtheorem{problem}{Problem}
\newtheorem{proposition}{Proposition}[section]
\newtheorem{lemma}{Lemma}[section]
\newtheorem{corollary}[theorem]{Corollary}
\newtheorem{example}{Example}[section]
\newtheorem{definition}[problem]{Definition}

\newcommand{\BEQA}{\begin{eqnarray}}
\newcommand{\EEQA}{\end{eqnarray}}
\newcommand{\define}{\stackrel{\triangle}{=}}
\bibliographystyle{IEEEtran}
\raggedbottom
\setlength{\parindent}{0pt}
\providecommand{\mbf}{\mathbf}
\providecommand{\pr}[1]{\ensuremath{\Pr\left(#1\right)}}
\providecommand{\qfunc}[1]{\ensuremath{Q\left(#1\right)}}
\providecommand{\sbrak}[1]{\ensuremath{{}\left[#1\right]}}
\providecommand{\lsbrak}[1]{\ensuremath{{}\left[#1\right.}}
\providecommand{\rsbrak}[1]{\ensuremath{{}\left.#1\right]}}
\providecommand{\brak}[1]{\ensuremath{\left(#1\right)}}
\providecommand{\lbrak}[1]{\ensuremath{\left(#1\right.}}
\providecommand{\rbrak}[1]{\ensuremath{\left.#1\right)}}
\providecommand{\cbrak}[1]{\ensuremath{\left\{#1\right\}}}
\providecommand{\lcbrak}[1]{\ensuremath{\left\{#1\right.}}
\providecommand{\rcbrak}[1]{\ensuremath{\left.#1\right\}}}
\theoremstyle{remark}
\newtheorem{rem}{Remark}
\newcommand{\sgn}{\mathop{\mathrm{sgn}}}
\providecommand{\abs}[1]{\left\vert#1\right\vert}
\providecommand{\res}[1]{\Res\displaylimits_{#1}} 
\providecommand{\norm}[1]{\left\lVert#1\right\rVert}
%\providecommand{\norm}[1]{\lVert#1\rVert}
\providecommand{\mtx}[1]{\mathbf{#1}}
\providecommand{\mean}[1]{E\left[ #1 \right]}
\providecommand{\fourier}{\overset{\mathcal{F}}{ \rightleftharpoons}}
%\providecommand{\hilbert}{\overset{\mathcal{H}}{ \rightleftharpoons}}
\providecommand{\system}{\overset{\mathcal{H}}{ \longleftrightarrow}}
	%\newcommand{\solution}[2]{\textbf{Solution:}{#1}}
\newcommand{\solution}{\noindent \textbf{Solution: }}
\newcommand{\cosec}{\,\text{cosec}\,}
\providecommand{\dec}[2]{\ensuremath{\overset{#1}{\underset{#2}{\gtrless}}}}
\newcommand{\myvec}[1]{\ensuremath{\begin{pmatrix}#1\end{pmatrix}}}
\newcommand{\mydet}[1]{\ensuremath{\begin{vmatrix}#1\end{vmatrix}}}
\numberwithin{equation}{subsection}

\makeatletter
\@addtoreset{figure}{problem}
\makeatother
\let\StandardTheFigure\thefigure
\let\vec\mathbf

\renewcommand{\thefigure}{\theproblem}

\def\putbox#1#2#3{\makebox[0in][l]{\makebox[#1][l]{}\raisebox{\baselineskip}[0in][0in]{\raisebox{#2}[0in][0in]{#3}}}}
     \def\rightbox#1{\makebox[0in][r]{#1}}
     \def\centbox#1{\makebox[0in]{#1}}
     \def\topbox#1{\raisebox{-\baselineskip}[0in][0in]{#1}}
     \def\midbox#1{\raisebox{-0.5\baselineskip}[0in][0in]{#1}}
\vspace{3cm}
\title{Assignment 10}
\author{Sachinkumar Dubey - EE20MTECH11009}
\maketitle
\newpage
\bigskip
\renewcommand{\thefigure}{\theenumi}
\renewcommand{\thetable}{\theenumi}
%
Download the latex-tikz codes from 
%
\begin{lstlisting}
https://github.com/sachinomdubey/Matrix-theory/Assignment10
\end{lstlisting}
\section{Problem}
(Hoffman/Page123/8) : 
%
If $F$ is a field and $h$ is a polynomial over $F$ of degree $\geq 1$,  show that the mapping $f \rightarrow f(h)$ is a one-one linear transformation of $F[x]$ into $F[x]$. Show that this transformation is an isomorphism of $F[x]$ onto $F[x]$ if and only if deg $h = 1$. 
\section{Solution}
Here, $F[x]$ is a set of polynomials over field $F$, written as:
\begin{align}
    F[x]&=\left \{\sum_{i=0}^\infty a_ix^i\quad \mid \quad a_i\in F\right \}
\end{align}
Let,
\begin{align}
    G(f)&=f(h) \label{eq1.1}
\end{align}
Thus, $G(f)$ is clearly a function from $F[x]$ into $F[x]$.\\
Now, we need to show that the function $G$ is one-one linear transformation. Let us first show that $G$ is a linear transformation:
\begin{align}
    \text{Let, }f,g \in F[x] \text{ and }\alpha \in F \nonumber
\end{align}
\begin{align}
    G(\alpha f+g)&=(\alpha f+g)(h) \nonumber\\
    &=(\alpha f)(h)+g(h)\nonumber \\
    &=\alpha f(h)+g(h)\nonumber \\
    &=\alpha G(f)+G(g) \label{eq2.3} 
\end{align}
From \eqref{eq2.3}, $G$ is a linear transformation. \\
%
For $G$ to be one-one linear transformation, it should map a set of linearly independent polynomials in $F(x)$ to another  set of linearly independent polynomials in $F(x)$. let us consider the following basis set for $F(x)$:
\begin{align}
    S&=\{f_0,f_1,f_2,f_3,f_4,\hdots\}
    \intertext{Where,}
    f_i&=x^i
\end{align}
Since, the set $S$ forms the basis for $F(x)$, the set $S$ is a set of linearly independent polynomials. Let us apply the transformation $G$ to set S, then we obtain another set $S'$ as:
\begin{align}
    S'&=\{f_0(h),f_1(h),f_2(h),f_3(h),f_4(h),\hdots\}
    \intertext{Where,}
    f_i&=x^i
\end{align}
Here, The degree of each polynomial in set $S'$ is distinct and given by $i\cdot$deg$(h)$. Thus, set $S'$ is also a set of linearly independent polynomials.\\ \\
\textbf{Conclusion:} $G$ will maps any arbitrary set $S_a$ of linearly independent polynomials in $F(x)$ to another set $S_a'$ of linearly independent polynomials in $F(x)$. (Since any arbitrary set $S_a$ can be written in terms of basis set $S$). Hence, \underline{G is one-one linear transformation}.\\\\
Now, Let us prove that $G$ is an isomorphism of $F(x)$ onto $F(x)$ if and only if deg$(h)=1$. \\
Let deg$(h)=1$, then $h$ can be written as:
\begin{align}
    h=a+bx, \quad \text{Where, }b \ne 0
\end{align}
Let us define $h'$ such that:
\begin{align}
    h'=\frac1bx-\frac ab
\end{align}
Let $G'$ be the linear transformation from $F(x)$ to $F(x)$ given by:
\begin{align}
    G’(f)=f\brak{\frac1bx-\frac ab}
\end{align}
It can be shown that $G'$ is inverse of $G$ as follow:
\begin{align}
    G(G'(f))&=G\brak{f\brak{\frac1bx-\frac ab}}\\
    &=f\brak{a\brak{\frac1ax-\frac ba}+b}\\
    &=f(x)
\end{align}
Similarly, 
\begin{align}
    G'(G(f))&=G'\brak{f\brak{ax+b}}\\
    &=f\brak{\frac1a\brak{ax+b}-\frac ba}\\
    &=f(x)
\end{align}
Thus, $G'$ is inverse of $G$. Therefore, $G$ is isomorphism and we can say: 
\begin{align}
    \boxed{\deg(h)=1 \implies \text{$G$ is isomorphism.}} \label{eq16}
\end{align}
Let deg$(h)>1$, then 
\begin{align}
    \deg f(h)&=\deg f\cdot \deg h \\
    \implies \deg f(h)&\geq 1\\
    \implies G(f)=f(h) &\ne x
\end{align}
This means the image of $G$ does not contain polynomials of degree one. Hence $G$ is not onto and therefore $G$ can not be an isomorphism. Thus we can write:
\begin{align}
    \boxed{\deg(h)>1 \implies \text{$G$ is not isomorphism.}} \label{eq20} 
\end{align}
From \eqref{eq16} and \eqref{eq20}, We can conclude:
\begin{align}
    \boxed{\text{$G$ is isomorphism.} \iff \deg(h)=1}  
\end{align}
\end{document}
