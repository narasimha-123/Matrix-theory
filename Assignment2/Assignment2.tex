\documentclass[journal,12pt,twocolumn]{IEEEtran}

\usepackage{setspace}
\usepackage{gensymb}

\singlespacing


\usepackage[cmex10]{amsmath}

\usepackage{amsthm}

\usepackage{mathrsfs}
\usepackage{txfonts}
\usepackage{stfloats}
\usepackage{bm}
\usepackage{cite}
\usepackage{cases}
\usepackage{subfig}

\usepackage{longtable}
\usepackage{multirow}

\usepackage{enumitem}
\usepackage{mathtools}
\usepackage{steinmetz}
\usepackage{tikz}
\usepackage{circuitikz}
\usepackage{verbatim}
\usepackage{tfrupee}
\usepackage[breaklinks=true]{hyperref}
\usepackage{graphicx}
\usepackage{tkz-euclide}

\usetikzlibrary{calc,math}
\usepackage{listings}
    \usepackage{color}                                            %%
    \usepackage{array}                                            %%
    \usepackage{longtable}                                        %%
    \usepackage{calc}                                             %%
    \usepackage{multirow}                                         %%
    \usepackage{hhline}                                           %%
    \usepackage{ifthen}                                           %%
    \usepackage{lscape}     
\usepackage{multicol}
\usepackage{chngcntr}

\DeclareMathOperator*{\Res}{Res}

\renewcommand\thesection{\arabic{section}}
\renewcommand\thesubsection{\thesection.\arabic{subsection}}
\renewcommand\thesubsubsection{\thesubsection.\arabic{subsubsection}}

\renewcommand\thesectiondis{\arabic{section}}
\renewcommand\thesubsectiondis{\thesectiondis.\arabic{subsection}}
\renewcommand\thesubsubsectiondis{\thesubsectiondis.\arabic{subsubsection}}


\hyphenation{op-tical net-works semi-conduc-tor}
\def\inputGnumericTable{}                                 %%

\lstset{
%language=C,
frame=single, 
breaklines=true,
columns=fullflexible
}
\begin{document}


\newtheorem{theorem}{Theorem}[section]
\newtheorem{problem}{Problem}
\newtheorem{proposition}{Proposition}[section]
\newtheorem{lemma}{Lemma}[section]
\newtheorem{corollary}[theorem]{Corollary}
\newtheorem{example}{Example}[section]
\newtheorem{definition}[problem]{Definition}

\newcommand{\BEQA}{\begin{eqnarray}}
\newcommand{\EEQA}{\end{eqnarray}}
\newcommand{\define}{\stackrel{\triangle}{=}}
\bibliographystyle{IEEEtran}

\providecommand{\mbf}{\mathbf}
\providecommand{\pr}[1]{\ensuremath{\Pr\left(#1\right)}}
\providecommand{\qfunc}[1]{\ensuremath{Q\left(#1\right)}}
\providecommand{\sbrak}[1]{\ensuremath{{}\left[#1\right]}}
\providecommand{\lsbrak}[1]{\ensuremath{{}\left[#1\right.}}
\providecommand{\rsbrak}[1]{\ensuremath{{}\left.#1\right]}}
\providecommand{\brak}[1]{\ensuremath{\left(#1\right)}}
\providecommand{\lbrak}[1]{\ensuremath{\left(#1\right.}}
\providecommand{\rbrak}[1]{\ensuremath{\left.#1\right)}}
\providecommand{\cbrak}[1]{\ensuremath{\left\{#1\right\}}}
\providecommand{\lcbrak}[1]{\ensuremath{\left\{#1\right.}}
\providecommand{\rcbrak}[1]{\ensuremath{\left.#1\right\}}}
\theoremstyle{remark}
\newtheorem{rem}{Remark}
\newcommand{\sgn}{\mathop{\mathrm{sgn}}}
\providecommand{\abs}[1]{\left\vert#1\right\vert}
\providecommand{\res}[1]{\Res\displaylimits_{#1}} 
\providecommand{\norm}[1]{\left\lVert#1\right\rVert}
%\providecommand{\norm}[1]{\lVert#1\rVert}
\providecommand{\mtx}[1]{\mathbf{#1}}
\providecommand{\mean}[1]{E\left[ #1 \right]}
\providecommand{\fourier}{\overset{\mathcal{F}}{ \rightleftharpoons}}
%\providecommand{\hilbert}{\overset{\mathcal{H}}{ \rightleftharpoons}}
\providecommand{\system}{\overset{\mathcal{H}}{ \longleftrightarrow}}
	%\newcommand{\solution}[2]{\textbf{Solution:}{#1}}
\newcommand{\solution}{\noindent \textbf{Solution: }}
\newcommand{\cosec}{\,\text{cosec}\,}
\providecommand{\dec}[2]{\ensuremath{\overset{#1}{\underset{#2}{\gtrless}}}}
\newcommand{\myvec}[1]{\ensuremath{\begin{pmatrix}#1\end{pmatrix}}}
\newcommand{\mydet}[1]{\ensuremath{\begin{vmatrix}#1\end{vmatrix}}}

\numberwithin{equation}{subsection}

\makeatletter
\@addtoreset{figure}{problem}
\makeatother
\let\StandardTheFigure\thefigure
\let\vec\mathbf

\renewcommand{\thefigure}{\theproblem}

\def\putbox#1#2#3{\makebox[0in][l]{\makebox[#1][l]{}\raisebox{\baselineskip}[0in][0in]{\raisebox{#2}[0in][0in]{#3}}}}
     \def\rightbox#1{\makebox[0in][r]{#1}}
     \def\centbox#1{\makebox[0in]{#1}}
     \def\topbox#1{\raisebox{-\baselineskip}[0in][0in]{#1}}
     \def\midbox#1{\raisebox{-0.5\baselineskip}[0in][0in]{#1}}
\vspace{3cm}
\title{Assignment 2}
\author{Sachinkumar Dubey}

\maketitle
\newpage

\bigskip
\renewcommand{\thefigure}{\theenumi}
\renewcommand{\thetable}{\theenumi}
Download all python codes from 
\begin{lstlisting}
https://github.com/sachinomdubey/Matrix-theory/Assignment2/codes
\end{lstlisting}
%
and latex-tikz codes from 
%
\begin{lstlisting}
https://github.com/sachinomdubey/Matrix-theory/Assignment2
\end{lstlisting}
%

\noindent Q no. 73. Find the angle between the following pair of lines: Also find the closest points and minimum distance between them.
\begin{enumerate}
\item
\begin{align}
L_1: \quad \vec{x} &= \myvec{2\\-5\\1} + \lambda_1\myvec{3 \\ 2 \\6}
\\
L_2: \quad \vec{x} &= \myvec{7\\-6\\0} + \lambda_2\myvec{1 \\ 2 \\2}
\end{align}
\item
\begin{align}
L_1: \quad \vec{x} &= \myvec{3\\1\\-2} + \lambda_1\myvec{1 \\ -1 \\-2}
\\
L_2: \quad \vec{x} &= \myvec{2\\-1\\-56} + \lambda_2\myvec{3 \\ -5 \\-4}
\end{align}
\end{enumerate}
%
\\
\solution 
\begin{enumerate}
\item The direction vectors of the lines are
\myvec{3 \\ 2 \\6} and \myvec{1 \\ 2 \\2}. \\
Thus, the angle $\theta$ between two vectors is given by 
%
\begin{align}
\label{eq:line_scalar_prod}
\cos \theta &= \frac{\vec{a}^T\vec{b}}{\norm{\vec{a}}\norm{\vec{b}}}
\\
&=\frac{19}{3\times7}
\\
\implies \theta &= 25.21\degree
\end{align}
\item The direction vectors of the lines are
\myvec{1 \\ -1 \\-2} and \myvec{3 \\ -5 \\-4}. \\
Thus, the angle $\theta$ between two vectors is given by 
%
\begin{align}
\label{eq:line_scalar_prod}
\cos \theta &= \frac{\vec{a}^T\vec{b}}{\norm{\vec{a}}\norm{\vec{b}}}
\\
&=\frac{16}{\sqrt6\times\sqrt50}
\\
\implies \theta &= 22.52\degree
\end{align}
\end{enumerate} 
\\
\textbf{Note :} In both problems, the respective pair of lines do not intersect each other (called skew lines), The obtained angle is the angle between the direction vectors of the lines. The proof that the pair of lines do not intersect in Problem 1 is as follows:\\
\\
\textbf{Problem 1 :} Equating the x, y and z components of both lines and forming equation in the augmented matrix form. The Matrix is row reduced as follows:
\begin{align}
\myvec{3 & -1 & 5\\2 & -2 & -1\\6 &-2&-1}\\
\xleftrightarrow[]{R_1\leftarrow R_1/3}
\myvec{3 & -1/3 & 5/3 \\2 & -2 & -1\\6 &-2&-1}\\
\xleftrightarrow[]{R_2\leftarrow R_2-2R_1}   
\myvec{3 & -1/3 & 5/3\\0 & -4/3 & -13/3\\6 &-2&-1}\\
\xleftrightarrow[]{R_3\leftarrow R_3-6R_1}
\myvec{3 & -1/3 & 5/3\\0 & -4/3 & -13/3\\0 &0&-11}
\end{align} 
Here, Rank(A) $\neq$ Rank(A$\mid$B). Hence, these three equations are inconsistent, which proves that the two lines do not intersect in the 3D plane. (Where A is the coefficient matrix and A$\mid$B is the augmented matrix.) \\
\\
\textbf{Finding the closest points on the lines and minimum distance (Problem 1) :}\\
The closest points (A and B) on skew lines are the points which form a vector such that the vector is perpendicular to both the lines as shown in Fig.\ref{Fig1}.
\\
\begin{figure}[h!]
\centering
\includegraphics[width=8cm, height=5cm]{Figure_3}
\caption{Condition for closest points}
\label{Fig1}
\end{figure}
\\
From the given equations, Points A and B are:
\\
\begin{align}
   Point-A = \myvec{2+3\lambda_1 \\ -5+2\lambda_1 \\ 1+6\lambda_1} \label{15} \\ 
   Point-B= \myvec{7+\lambda_2 \\ -6+2\lambda_2 \\ 2\lambda_2} \label{16} 
\end{align}
\\
Thus, The vector \vec{AB} is given by,
\begin{align}
   \vec{AB}= \myvec{5-3\lambda_1+\lambda_2 \\ -1-2\lambda_1+2\lambda_2\\ -1-6\lambda_1+2\lambda_2}
\end{align}
\\
Since, vector \vec{AB} is perpendicular to both lines, the dot product of vector \vec{AB} with the respective directional vectors should be zero. \\
\\
Performing the dot product for line-1 and equating to 0, we get:
\begin{align}
\myvec{3 & 2 & 6}
\myvec{5-3\lambda_1+\lambda_2 \\ -1-2\lambda_1+2\lambda_2\\ -1-6\lambda_1+2\lambda_2} = 0 \\
\implies \myvec{-49 & 19}\myvec{\lambda_1 \\ \lambda_2}=-7 \label{19}
\end{align}
\newpage \noindent
Performing the dot product for line-2 and equating to 0, we get:
\begin{align}
\myvec{1 & 2 & 2}
\myvec{5-3\lambda_1+\lambda_2 \\ -1-2\lambda_1+2\lambda_2\\ -1-6\lambda_1+2\lambda_2} = 0 \\
\implies \myvec{-19 & 9}\myvec{\lambda_1 \\ \lambda_2}=-1 \label{21}
\end{align}
Writing equations \ref{19} and \ref{21} in augmented matrix form and reducing the rows :\\
\begin{align}
\myvec{-49 & 19 & -7\\-19 & 9 & -1}\\
\xleftrightarrow[]{R_1\leftarrow R_1\times-1/49}
\myvec{1 & -19/49 & 1/7\\-19 & 9 & -1}\\
\xleftrightarrow[]{R_2\leftarrow R_2+19R_1}
\myvec{1 & -19/49 & 1/7\\0 & 80/49 & 12/7}\\
\xleftrightarrow[]{R_2\leftarrow \frac{49}{80}\times R_2}
\myvec{1 & -19/49 & 1/7\\0 & 1 & 21/20}\\
\xleftrightarrow[]{R_1\leftarrow R_1+\frac{19}{49}\times R_2}
\myvec{1 & 0 & 11/20\\0 & 1 & 21/20}
\end{align}
\\
Thus, $\lambda_1$=11/20 and $\lambda_2$=21/20. Substituting values of $\lambda_1$ and $\lambda_2$ in equations \ref{15} and \ref{16}, we get the closest points as :\\
\begin{align}
 Point-A = \myvec{3.65 \\ -3.9\\ 4.3}\\ 
   Point-B= \myvec{8.05\\ -3.9\\ 2.1} 
\end{align} 
The minimum distance calculated from these two points is \textbf{4.92} units.

\newpage
\begin{figure}[h!]
\centering
\includegraphics[width=10cm, height=8cm]{Figure_1}
\caption{Problem 1 : Lines crossing each other, but not intersecting}
\label{Fig2}
\end{figure}
\\
The proof that the pair of lines do not intersect in Problem 1 is as follows:\\
\\
\textbf{Problem 2 :} Equating the x, y and z components of both lines and forming equation in the augmented matrix forms. The Matrix is row reduced as follows:\\
\begin{align}
\myvec{1 & -3 & -1\\-1 & 5 & -2\\-2 &4&-54}\\
\xleftrightarrow[]{R_2\leftarrow R_2+R_1}
\myvec{1 & -3 & -1 \\0 & 2 & -3\\-2 &4&-54}\\
\xleftrightarrow[]{R_3\leftarrow R_3+2R_1}   
\myvec{1 & -3 & -1 \\0 & 2 & -3\\0 &-2&-56}\\
\xleftrightarrow[]{R_2\leftarrow R_2/2}
\myvec{1 & -3 & -1 \\0 & 1 & -3/2\\0 &-2&-56}\\
\xleftrightarrow[]{R_3\leftarrow R_3+2R_2}
\myvec{1 & -3 & -1 \\0 & 1 & -3/2\\0 &0&-59}
\end{align} 
Here, Rank(A) $\neq$ Rank(A$\mid$B).
\\
Hence, the equations are inconsistent, which proves that the two lines do not intersect in the 3D plane. (Where A is the coefficient matrix and A$\mid$B is the augmented matrix.) \\
\newpage \noindent
\textbf{Finding the closest points on the lines and minimum distance (Problem 2) :}\\
The closest points (A and B) on skew lines are the points which form a vector such that the vector is perpendicular to both the lines as shown in Fig.\ref{Fig3}.
\\
\begin{figure}[h!]
\centering
\includegraphics[width=8cm, height=5cm]{Figure_3}
\caption{Condition for closest points}
\label{Fig3}
\end{figure}
\\
From the given equations, Points A and B are:
\\
\begin{align}
   Point-A = \myvec{3+\lambda_1 \\ 1-\lambda_1 \\ -2-2\lambda_1} \label{34} \\ 
   Point-B= \myvec{2+3\lambda_2 \\ -1-5\lambda_2 \\ -56-4\lambda_2} \label{35} 
\end{align}
\\
Thus, The vector \vec{AB} is given by,
\begin{align}
   \vec{AB}= \myvec{-1-\lambda_1+3\lambda_2 \\ -2+\lambda_1-5\lambda_2\\ -54+2\lambda_1-4\lambda_2}
\end{align}
\\
Since, vector \vec{AB} is perpendicular to both lines, the dot product of vector \vec{AB} with the respective directional vectors should be zero. \\
\\
Performing the dot product for line-1 and equating to 0, we get:
\begin{align}
\myvec{1 & -1 & -2}
\myvec{-1-\lambda_1+3\lambda_2 \\ -2+\lambda_1-5\lambda_2\\ -54+2\lambda_1-4\lambda_2}=0\\
\implies \myvec{-6 & 16}\myvec{\lambda_1 \\ \lambda_2}=-109 \label{38}
\end{align}
\newpage \noindent
Performing the dot product for line-2 and equating to 0, we get:
\begin{align}
\myvec{3 & -5 & -4}
\myvec{-1-\lambda_1+3\lambda_2 \\ -2+\lambda_1-5\lambda_2\\ -54+2\lambda_1-4\lambda_2}=0 \\
\implies \myvec{-16 & 50}\myvec{\lambda_1 \\ \lambda_2}=-223 \label{40}
\end{align}
Writing equations \ref{38} and \ref{40} in augmented matrix form and reducing the rows :\\
\begin{align}
\myvec{-6 & 16 & -109\\-16 & 50 & -223}\\
\xleftrightarrow[]{R_1\leftarrow R_1\times-1/6}
\myvec{1 & -8/3 & 109/6\\-16 & 50 & -223}\\
\xleftrightarrow[]{R_2\leftarrow R_2+16R_1}
\myvec{1 & -8/3 & 109/6\\0 & 22/3 & 203/3}\\
\xleftrightarrow[]{R_2\leftarrow \frac{3}{22}\times R_2}
\myvec{1 & -8/3 & 109/6\\0 & 1 & 203/22}\\
\xleftrightarrow[]{R_1\leftarrow R_1+\frac{8}{3}\times R_2}
\myvec{1 & 0 & 941/22\\0 & 1 & 203/22}
\end{align}
\\
Thus, $\lambda_1$=42.77 and $\lambda_2$=9.23. Substituting values of $\lambda_1$ and $\lambda_2$ in equations \ref{34} and \ref{35}, we get the closest points as :\\
\begin{align}
 Point-A = \myvec{45.77 \\ -41.77\\ -87.54}\\ 
   Point-B= \myvec{29.69\\ -47.15\\ -92.92} 
\end{align} 
The minimum distance calculated from these two points is \textbf{17.79} units.
\newpage
\begin{figure}[h!]
\centering
\includegraphics[width=10cm, height=8cm]{Figure_2}
\caption{Problem 2 : Lines crossing each other, but not intersecting}
\label{Fig4}
\end{figure}
\end{document}