\documentclass[journal,12pt,twocolumn]{IEEEtran}

\usepackage{setspace}
\usepackage{gensymb}
\singlespacing
\usepackage[cmex10]{amsmath}

\usepackage{amsthm}

\usepackage{mathrsfs}
\usepackage{txfonts}
\usepackage{stfloats}
\usepackage{bm}
\usepackage{cite}
\usepackage{cases}
\usepackage{subfig}

\usepackage{longtable}
\usepackage{multirow}

\usepackage{enumitem}
\usepackage{mathtools}
\usepackage{steinmetz}
\usepackage{tikz}
\usepackage{circuitikz}
\usepackage{verbatim}
\usepackage{tfrupee}
\usepackage[breaklinks=true]{hyperref}
\usepackage{graphicx}
\usepackage{tkz-euclide}

\usetikzlibrary{calc,math}
\usepackage{listings}
    \usepackage{color}                                            %%
    \usepackage{array}                                            %%
    \usepackage{longtable}                                        %%
    \usepackage{calc}                                             %%
    \usepackage{multirow}                                         %%
    \usepackage{hhline}                                           %%
    \usepackage{ifthen}                                           %%
    \usepackage{lscape}     
\usepackage{multicol}
\usepackage{chngcntr}

\DeclareMathOperator*{\Res}{Res}

\renewcommand\thesection{\arabic{section}}
\renewcommand\thesubsection{\thesection.\arabic{subsection}}
\renewcommand\thesubsubsection{\thesubsection.\arabic{subsubsection}}

\renewcommand\thesectiondis{\arabic{section}}
\renewcommand\thesubsectiondis{\thesectiondis.\arabic{subsection}}
\renewcommand\thesubsubsectiondis{\thesubsectiondis.\arabic{subsubsection}}


\hyphenation{op-tical net-works semi-conduc-tor}
\def\inputGnumericTable{}                                 %%

\lstset{
%language=C,
frame=single, 
breaklines=true,
columns=fullflexible
}
\begin{document}


\newtheorem{theorem}{Theorem}[section]
\newtheorem{problem}{Problem}
\newtheorem{proposition}{Proposition}[section]
\newtheorem{lemma}{Lemma}[section]
\newtheorem{corollary}[theorem]{Corollary}
\newtheorem{example}{Example}[section]
\newtheorem{definition}[problem]{Definition}

\newcommand{\BEQA}{\begin{eqnarray}}
\newcommand{\EEQA}{\end{eqnarray}}
\newcommand{\define}{\stackrel{\triangle}{=}}
\bibliographystyle{IEEEtran}
\raggedbottom
\setlength{\parindent}{0pt}
\providecommand{\mbf}{\mathbf}
\providecommand{\pr}[1]{\ensuremath{\Pr\left(#1\right)}}
\providecommand{\qfunc}[1]{\ensuremath{Q\left(#1\right)}}
\providecommand{\sbrak}[1]{\ensuremath{{}\left[#1\right]}}
\providecommand{\lsbrak}[1]{\ensuremath{{}\left[#1\right.}}
\providecommand{\rsbrak}[1]{\ensuremath{{}\left.#1\right]}}
\providecommand{\brak}[1]{\ensuremath{\left(#1\right)}}
\providecommand{\lbrak}[1]{\ensuremath{\left(#1\right.}}
\providecommand{\rbrak}[1]{\ensuremath{\left.#1\right)}}
\providecommand{\cbrak}[1]{\ensuremath{\left\{#1\right\}}}
\providecommand{\lcbrak}[1]{\ensuremath{\left\{#1\right.}}
\providecommand{\rcbrak}[1]{\ensuremath{\left.#1\right\}}}
\theoremstyle{remark}
\newtheorem{rem}{Remark}
\newcommand{\sgn}{\mathop{\mathrm{sgn}}}
\providecommand{\abs}[1]{\left\vert#1\right\vert}
\providecommand{\res}[1]{\Res\displaylimits_{#1}} 
\providecommand{\norm}[1]{\left\lVert#1\right\rVert}
%\providecommand{\norm}[1]{\lVert#1\rVert}
\providecommand{\mtx}[1]{\mathbf{#1}}
\providecommand{\mean}[1]{E\left[ #1 \right]}
\providecommand{\fourier}{\overset{\mathcal{F}}{ \rightleftharpoons}}
%\providecommand{\hilbert}{\overset{\mathcal{H}}{ \rightleftharpoons}}
\providecommand{\system}{\overset{\mathcal{H}}{ \longleftrightarrow}}
	%\newcommand{\solution}[2]{\textbf{Solution:}{#1}}
\newcommand{\solution}{\noindent \textbf{Solution: }}
\newcommand{\cosec}{\,\text{cosec}\,}
\providecommand{\dec}[2]{\ensuremath{\overset{#1}{\underset{#2}{\gtrless}}}}
\newcommand{\myvec}[1]{\ensuremath{\begin{pmatrix}#1\end{pmatrix}}}
\newcommand{\mydet}[1]{\ensuremath{\begin{vmatrix}#1\end{vmatrix}}}
\numberwithin{equation}{subsection}

\makeatletter
\@addtoreset{figure}{problem}
\makeatother
\let\StandardTheFigure\thefigure
\let\vec\mathbf

\renewcommand{\thefigure}{\theproblem}

\def\putbox#1#2#3{\makebox[0in][l]{\makebox[#1][l]{}\raisebox{\baselineskip}[0in][0in]{\raisebox{#2}[0in][0in]{#3}}}}
     \def\rightbox#1{\makebox[0in][r]{#1}}
     \def\centbox#1{\makebox[0in]{#1}}
     \def\topbox#1{\raisebox{-\baselineskip}[0in][0in]{#1}}
     \def\midbox#1{\raisebox{-0.5\baselineskip}[0in][0in]{#1}}
\vspace{3cm}
\title{Assignment 7}
\author{Sachinkumar Dubey - EE20MTECH11009}
\maketitle
\newpage
\bigskip
\renewcommand{\thefigure}{\theenumi}
\renewcommand{\thetable}{\theenumi}
Download all python codes from 
\begin{lstlisting}
https://github.com/sachinomdubey/Matrix-theory/Assignment7/codes
\end{lstlisting}
%
and latex-tikz codes from 
%
\begin{lstlisting}
https://github.com/sachinomdubey/Matrix-theory/Assignment7
\end{lstlisting}
\section{Problem}
(Rams 3.4.1) find the QR decomposition of the following:\\
\begin{align}
\vec{V}=\myvec{1 & 0 \\0 & -1}\label{eq:2.0.3}
\end{align}
\section{Solution}
Here, let the column vectors of $\vec{V}$ be $\alpha$ and $\beta$:
\begin{align}
\alpha=\myvec{1 \\ 0}\\
\beta=\myvec{0 \\-1}
\end{align}
To find $\vec{Q}=\myvec{\vec{u}_1 &\vec{u}_2}$, we will orthonormalise the columns of $\vec{V}$ using Gram Schmit method:
\begin{align}
\vec{u}_1=\frac{\alpha}{k_1}\\
k_1=\norm{\alpha}=\sqrt{1^2+0^2}=1\\
\implies \vec{u}_1=\frac{1}{1}\myvec{1 \\ 0}=\myvec{1 \\ 0}
\end{align}
\begin{align}
\vec{u}_2=\frac{\beta-r_1\vec{u}_1}{\norm{\beta-r_1\vec{u}_1}}\\
r_1=\frac{\vec{u}_1^T \beta}{\norm{\vec{u}_1}^2}=\frac{\myvec{1 & 0}\myvec{0 \\-1}}{\sqrt{0^2+(-1)^2}}=0\\
\implies \vec{u}_2=\frac{\beta}{\norm{\beta}}=\myvec{0 \\ -1}
\end{align}
\begin{align}
    \text{Also,} \quad k_2=\vec{u}_2^T \beta =\myvec{0 &-1}\myvec{0 \\-1} = 1
\end{align}
The QR decomposition is given as:
\begin{align}
    \myvec{\alpha & \beta}=\myvec{\vec{u}_1 &\vec{u}_2}\myvec{k_1 & r_1 \\0 & k_2} \label{eq2.0.12}
    \intertext{Where,}
    \vec{Q}=\myvec{\vec{u}_1 &\vec{u}_2}\\
    \vec{R}=\myvec{k_1 & r_1 \\0 & k_2}
\end{align}
Putting the values of $\vec{u}_1$, $\vec{u}_2$, $k_1$, $k_2$ and $r_1$ in equation \eqref{eq2.0.12}:
\begin{align}
    \boxed{\vec{V}=\myvec{1 & 0 \\0 & -1}\myvec{1 & 0 \\0 & 1}}
    \intertext{Where,}
    \vec{Q}=\myvec{1 & 0 \\0 & -1}\\
    \vec{R}=\myvec{1 & 0 \\0 & 1}
\end{align}
\textbf{NOTE:} Here, Matrix $\vec{V}$ already had orthonormal column vectors. Hence, The obtained $\vec{Q}$ by Gram schmit method is same as $\vec{V}$ and the obtained $\vec{R}$ is an identity matrix.
\end{document}