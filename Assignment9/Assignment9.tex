\documentclass[journal,12pt,twocolumn]{IEEEtran}

\usepackage{setspace}
\usepackage{gensymb}
\singlespacing
\usepackage[cmex10]{amsmath}

\usepackage{amsthm}

\usepackage{mathrsfs}
\usepackage{txfonts}
\usepackage{stfloats}
\usepackage{bm}
\usepackage{cite}
\usepackage{cases}
\usepackage{subfig}

\usepackage{longtable}
\usepackage{multirow}

\usepackage{enumitem}
\usepackage{mathtools}
\usepackage{steinmetz}
\usepackage{tikz}
\usepackage{circuitikz}
\usepackage{verbatim}
\usepackage{tfrupee}
\usepackage[breaklinks=true]{hyperref}
\usepackage{graphicx}
\usepackage{tkz-euclide}

\usetikzlibrary{calc,math}
\usepackage{listings}
    \usepackage{color}                                            %%
    \usepackage{array}                                            %%
    \usepackage{longtable}                                        %%
    \usepackage{calc}                                             %%
    \usepackage{multirow}                                         %%
    \usepackage{hhline}                                           %%
    \usepackage{ifthen}                                           %%
    \usepackage{lscape}     
\usepackage{multicol}
\usepackage{chngcntr}

\DeclareMathOperator*{\Res}{Res}

\renewcommand\thesection{\arabic{section}}
\renewcommand\thesubsection{\thesection.\arabic{subsection}}
\renewcommand\thesubsubsection{\thesubsection.\arabic{subsubsection}}

\renewcommand\thesectiondis{\arabic{section}}
\renewcommand\thesubsectiondis{\thesectiondis.\arabic{subsection}}
\renewcommand\thesubsubsectiondis{\thesubsectiondis.\arabic{subsubsection}}


\hyphenation{op-tical net-works semi-conduc-tor}
\def\inputGnumericTable{}                                 %%

\lstset{
%language=C,
frame=single, 
breaklines=true,
columns=fullflexible
}
\begin{document}


\newtheorem{theorem}{Theorem}[section]
\newtheorem{problem}{Problem}
\newtheorem{proposition}{Proposition}[section]
\newtheorem{lemma}{Lemma}[section]
\newtheorem{corollary}[theorem]{Corollary}
\newtheorem{example}{Example}[section]
\newtheorem{definition}[problem]{Definition}

\newcommand{\BEQA}{\begin{eqnarray}}
\newcommand{\EEQA}{\end{eqnarray}}
\newcommand{\define}{\stackrel{\triangle}{=}}
\bibliographystyle{IEEEtran}
\raggedbottom
\setlength{\parindent}{0pt}
\providecommand{\mbf}{\mathbf}
\providecommand{\pr}[1]{\ensuremath{\Pr\left(#1\right)}}
\providecommand{\qfunc}[1]{\ensuremath{Q\left(#1\right)}}
\providecommand{\sbrak}[1]{\ensuremath{{}\left[#1\right]}}
\providecommand{\lsbrak}[1]{\ensuremath{{}\left[#1\right.}}
\providecommand{\rsbrak}[1]{\ensuremath{{}\left.#1\right]}}
\providecommand{\brak}[1]{\ensuremath{\left(#1\right)}}
\providecommand{\lbrak}[1]{\ensuremath{\left(#1\right.}}
\providecommand{\rbrak}[1]{\ensuremath{\left.#1\right)}}
\providecommand{\cbrak}[1]{\ensuremath{\left\{#1\right\}}}
\providecommand{\lcbrak}[1]{\ensuremath{\left\{#1\right.}}
\providecommand{\rcbrak}[1]{\ensuremath{\left.#1\right\}}}
\theoremstyle{remark}
\newtheorem{rem}{Remark}
\newcommand{\sgn}{\mathop{\mathrm{sgn}}}
\providecommand{\abs}[1]{\left\vert#1\right\vert}
\providecommand{\res}[1]{\Res\displaylimits_{#1}} 
\providecommand{\norm}[1]{\left\lVert#1\right\rVert}
%\providecommand{\norm}[1]{\lVert#1\rVert}
\providecommand{\mtx}[1]{\mathbf{#1}}
\providecommand{\mean}[1]{E\left[ #1 \right]}
\providecommand{\fourier}{\overset{\mathcal{F}}{ \rightleftharpoons}}
%\providecommand{\hilbert}{\overset{\mathcal{H}}{ \rightleftharpoons}}
\providecommand{\system}{\overset{\mathcal{H}}{ \longleftrightarrow}}
	%\newcommand{\solution}[2]{\textbf{Solution:}{#1}}
\newcommand{\solution}{\noindent \textbf{Solution: }}
\newcommand{\cosec}{\,\text{cosec}\,}
\providecommand{\dec}[2]{\ensuremath{\overset{#1}{\underset{#2}{\gtrless}}}}
\newcommand{\myvec}[1]{\ensuremath{\begin{pmatrix}#1\end{pmatrix}}}
\newcommand{\mydet}[1]{\ensuremath{\begin{vmatrix}#1\end{vmatrix}}}
\numberwithin{equation}{subsection}

\makeatletter
\@addtoreset{figure}{problem}
\makeatother
\let\StandardTheFigure\thefigure
\let\vec\mathbf

\renewcommand{\thefigure}{\theproblem}

\def\putbox#1#2#3{\makebox[0in][l]{\makebox[#1][l]{}\raisebox{\baselineskip}[0in][0in]{\raisebox{#2}[0in][0in]{#3}}}}
     \def\rightbox#1{\makebox[0in][r]{#1}}
     \def\centbox#1{\makebox[0in]{#1}}
     \def\topbox#1{\raisebox{-\baselineskip}[0in][0in]{#1}}
     \def\midbox#1{\raisebox{-0.5\baselineskip}[0in][0in]{#1}}
\vspace{3cm}
\title{Assignment 9}
\author{Sachinkumar Dubey - EE20MTECH11009}
\maketitle
\newpage
\bigskip
\renewcommand{\thefigure}{\theenumi}
\renewcommand{\thetable}{\theenumi}
Download all python codes from 
\begin{lstlisting}
https://github.com/sachinomdubey/Matrix-theory/Assignment9/codes
\end{lstlisting}
%
and latex-tikz codes from 
%
\begin{lstlisting}
https://github.com/sachinomdubey/Matrix-theory/Assignment9
\end{lstlisting}
\section{Problem}
(Hoffman/Page27/12)\\ Prove that the given matrix is invertible and $\vec{A}^{-1}$ has integer values.
\begin{align}
    \vec{A}=\myvec{1 & \tfrac{1}{2} &.&.&\tfrac{1}{n}\\
    \tfrac{1}{2} & \tfrac{1}{3} &.&.&\tfrac{1}{n+1}\\
    .&.&.&.&.\\.&.&.&.&.\\\tfrac{1}{n} & \tfrac{1}{n+1} &.&.&\tfrac{1}{2n-1}} \label{eq1}
\end{align}
\section{Solution}
\underline{\textbf{Proof that A is Invertible.}}\\
The elements of the given matrix is of the form:
\begin{align}
    A_{ij} = \frac{1}{i+j-1}= \int_{0}^{1}t^{i+j-2}dt= \int_{0}^{1}t^{i-1}t^{j-1}dt \label{eq2}
\end{align}
We will prove that the matrix $\vec{A}$ is positive definite:\\
$\vec{A}$ is Positive definite, if $\vec{X}\Vec{A}\vec{X}^T>0$
\begin{align}
   \text{Let } \vec{X}=(x_i)_{1\leq i\leq n} \in \mathbb{R}^N\\
   \vec{X}\Vec{A}\vec{X}^T=\sum_{1\leq i,j\leq n}\frac{x_ix_j}{i+j-1}
\end{align}
From \eqref{eq2},
\begin{align}
   \vec{X}\Vec{A}\vec{X}^T=\sum_{1\leq i,j\leq n}x_ix_j\int_0^1t^{i+j-2}dt\\
   \vec{X}\Vec{A}\vec{X}^T=\int_0^1\left(\sum_{i=1}^nx_it^{i-1}\right)\left(\sum_{j=1}^nx_jt^{j-1}\right)dt
\end{align}
\begin{align}
       \implies \vec{X}\Vec{A}\vec{X}^T=\int_0^1\left(\sum_{i=1}^nx_it^{i-1}\right)^2dt>0
\end{align}
Thus, Matrix $\vec{A}$ is Positive definite. \\
Now, let's say $\lambda$ is an eigen value of $\vec{A}$. Then, for the corresponding eigen vector $\vec{X}=(x_1,x_2,...,x_n)$, we can write:
\begin{align}
       \vec{X}\Vec{A}\vec{X}^T=\vec{X}\lambda\vec{X}^T \quad [\because  \Vec{A}\vec{X}^T=\lambda\vec{X}^T]\\
       \implies \vec{X}\Vec{A}\vec{X}^T=\norm{\vec{X}}^2\lambda\\
       \implies \lambda=\frac{\vec{X}\Vec{A}\vec{X}^T}{\norm{\vec{X}}^2}>0
\end{align}
So, all of the eigenvalues belonging to  $\vec{A}$ must be positive. The product of the eigenvalues of a matrix equals the determinant.
\begin{align}
    \boxed{\therefore \det({\vec{A}})>0}
\end{align}
Thus, the given matrix $\vec{A}$ is non-singular and its inverse exist (Invertible).\\ \\
\underline{\textbf{Proof that $\vec{A}^{-1}$ has integer values.}}\\
Let us consider a set of shifted legendre polynomials as follow:
\begin{align}
    P_i(x)=
    \left\{\begin{matrix}
P_1(x)=P_{11}\\ 
P_2(x)=P_{21}+P_{22}x\\
P_3(x)=P_{31}+P_{32}x+p_{33}x^2\\
.\\.\\
P_n(x)=P_{n1}+P_{n2}x+P_{n3}x^2+..+P_{nn}x^{n-1}
\end{matrix}\right. \label{eq11} \\
\intertext{Where, the coefficients $P_{ij}$ are given as:}
P_{ij}=(-1)^{i+j-1} {j-1 \choose i-1} {i+j-2 \choose i-1} \label{eq12}
\end{align}
The shifted legendre polynomials are analogous to legendre polynomials, but defined on the interval $[0,1]$ (whereas the interval is $[-1,1]$ for legendre polynomial). \newpage
A set of shifted legendre polynomials obey the following orthogonal relationship:
\begin{align}
    \int_{0}^{1}P_i(x)P_j(x)dx=0\text{ for }i \neq j\\
    \int_{0}^{1}P_i(x)P_j(x)dx=\frac{1}{2i+1}\text{ for }i=j \label{eq14}
\end{align}
Forming a matrix $\vec{P}$ whose elements are the coefficients of polynomials in \eqref{eq11}
\begin{align}
    \vec{P}=\myvec{P_{11} & 0 &.&.&0\\
    P_{21} & P_{22} &.&.&0\\
    .&.&.&.&.\\.&.&.&.&.\\P_{n1} & P_{n2} &.&.&P_{nn}}
\end{align}
Forming a matrix $\vec{P}\vec{A}\vec{P}^T$, the elements of the matrix $\vec{P}\vec{A}\vec{P}^T$ can be written as:
\begin{align}
    \vec{P}\vec{A}\vec{P}^T_{ij}=\sum_{s=1}^N\sum_{r=1}^NP_{ir}P_{js}A_{rs} \label{eq15}
    \\
    \intertext{From \eqref{eq2} with suitable change in variable notations, we can write:}
    A_{rs}=\int_0^1x^{r-1}x^{s-1}dx \label{eq16}\\
    \intertext{From \eqref{eq15} and \eqref{eq16},}
    \vec{P}\vec{A}\vec{P}^T_{ij}=\int_0^1\sum_{s=1}^N\sum_{r=1}^NP_{ir}P_{js}x^{r-1}x^{s-1}dx\\
    \vec{P}\vec{A}\vec{P}^T_{ij}=\int_0^1\sum_{s=1}^NP_{ir}x^{r-1}\sum_{r=1}^NP_{js}x^{s-1}dx \\
    \vec{P}\vec{A}\vec{P}^T_{ij}=\int_0^1P_i(x)P_j(x)dx\\
    \intertext{From \eqref{eq14}}
    \vec{P}\vec{A}\vec{P}^T_{ij}=
    \left\{\begin{matrix}
    0 & i\neq j\\ 
    \frac{1}{2i+1} & i=j
    \end{matrix}\right.
\end{align}
Thus, Matrix $\vec{P}\vec{A}\vec{P}^T$ is diagonal matrix:
\begin{align}
    \vec{P}\vec{A}\vec{P}^T=\vec{D}=\myvec{\frac{1}{3} & 0 &.&.&0\\
    0 & \frac{1}{5} &.&.&0\\
    .&.&.&.&.\\.&.&.&.&.\\0 &0 &.&.&\frac{1}{2n+1}} \label{eq22}
\end{align}
From \eqref{eq22}, the inverse of matrix $\vec{A}$ can be written as:
\begin{align}
    \vec{A}=\vec{P}^{-1}\vec{D}(\vec{P}^T)^{-1}\\
    \implies \vec{A}^{-1}=\vec{P}^T\vec{D}^{-1}\vec{P}
\end{align}
From \eqref{eq12} and \eqref{eq22}, It can be clearly observed that the elements of matrix $\vec{P}$, $\vec{P}^T$ and $\vec{D}^{-1}$ are all integers given as:
\begin{align}
    \vec{P}_{ij}=(-1)^{i+j-1} {j-1 \choose i-1} {i+j-2 \choose i-1} \\
    \vec{D}^{-1}_{ij}=\frac{1}{\vec{D}_{ij}}=   \left\{\begin{matrix}
    0 & i\neq j\\ 
    2i+1 & i=j
    \end{matrix}\right.
\end{align}
Since, matrix $\vec{P}$, $\vec{P}^T$ and $\vec{D}^{-1}$ are integer matrices, therefore $\vec{A}^{-1}$ is also an integer matrix.\\
Hence proved.
\\ \\ 
\textbf{Observations:} 
\begin{enumerate}
    \item The given matrix is a $n\times n$ Hilbert matrix. Which is always invertible with its inverse having integer values.
    \item The Hilbert matrix is symmetric and positive definite.
\end{enumerate}
\end{document}